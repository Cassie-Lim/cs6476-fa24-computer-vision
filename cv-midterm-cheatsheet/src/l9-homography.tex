\section{Homography and Image Mosaics}
1. \textbf{Goal:} Use homography to align and stitch images into a seamless mosaic. \\
2. \textbf{Homography:} A projective transformation that maps points from one image plane to another. \\
   % - Allows mapping of rectangles to quadrilaterals and preserves straight lines (not necessarily parallel lines). \\
   - Preserve straight lines, not necessarily parallel lines/length/angle. \\
   - Represented by a 3x3 matrix \(H\):
   \[
   \mathbf{p'} = H \cdot \mathbf{p}
   \]
   % where \(\mathbf{p}\) and \(\mathbf{p'}\) are homogeneous coordinates of corresponding points in two images.

\subsection*{Generating Image Mosaics}
1. Capture a sequence of images from the same camera position. \\
2. Compute the transformation between consecutive images w feature-based alignment. \\
3. Use homography to transform and align images. \\
4. Blend aligned images to create the final mosaic. \\
% 5. Repeat for additional images to extend the mosaic.

\subsection*{Image Warping and Reprojection}
1. \textbf{Forward Warping:} 
% Map each pixel in the source image to a new position:
   \[
   (x', y') = T(x, y)
   \]
   - If a pixel lands between two pixels in the target image, distribute its color among neighbors (splatting). \\
2. \textbf{Inverse Warping:} 
% Compute the source pixel for each target pixel using the inverse transformation:
   \[
   (x, y) = T^{-1}(x', y')
   \]
   - Interpolate color values from neighbors (e.g., nearest neighbor, bilinear interpolation).

\subsection*{RANSAC for Homography Estimation}
1. Randomly select 4 pairs of corresponding pts. \\
2. Compute the homography matrix \(H\). \\
3. Identify inliers—pairs that satisfy:
   \[
   \text{SSD}(\mathbf{p'}, H \cdot \mathbf{p}) < \epsilon
   \]
4. Keep the set of inliers w the largest size. \\
5. Recompute \(H\) using all inliers w least squares.

% \subsection*{Applications of Homography}
% 1. \textbf{Image Mosaics:} Stitch images to create panoramic views. \\
% 2. \textbf{Image Rectification:} Align images to correct geometric distortions. \\
% 3. \textbf{Virtual Views:} Synthesize new viewpoints using homography and image warping.

% \subsection*{Summary of Alignment and Warping}
% 1. Write 2D transformations as matrix-vector multiplication. \\
% 2. Use homogeneous coordinates to handle translations. \\
% 3. Perform image warping using forward or inverse mapping. \\
% 4. Fit transformations by solving for unknown parameters using corresponding points. \\
% 5. Create mosaics by warping and stitching images aligned by homography.
