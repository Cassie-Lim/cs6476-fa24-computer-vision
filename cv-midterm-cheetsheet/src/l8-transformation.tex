\section{Fitting a 2D Transformation}
% 1. Given two images, the goal is to learn a transformation \(T\) that aligns them. \\
% 2. Transformation \(T\) can be fit using matching feature pairs called correspondences. \\

\subsection*{Parametric Warping}
1. Transformation \(T\) can take different forms: \\
   - Translation, Rotation, Scaling, Affine, and Perspective. \\
2. These transformations can be represented as matrix operations, such as:
   \[
   \mathbf{p'} = \mathbf{T} \cdot \mathbf{p}
   \]
   % where \(\mathbf{p} = (x, y)\) and \(\mathbf{p'} = (x', y')\) are points in the original and transformed space.

\subsection*{Basic Transformations as Matrices}
1. \textbf{Scaling}, \quad \textbf{Rotation}: \\
   \[
   \mathbf{S} = 
   \begin{bmatrix}
   s_x & 0 \\
   0 & s_y
   \end{bmatrix}, \quad
   \mathbf{R} =
   \begin{bmatrix}
   \cos \theta & -\sin \theta \\
   \sin \theta & \cos \theta
   \end{bmatrix}
   \] 
2. \textbf{Shear}, \quad \textbf{Translation}: \\
   \[
   \mathbf{S} = 
   \begin{bmatrix}
   1 & \alpha \\
   \beta & 1
   \end{bmatrix}, \quad
   \mathbf{T} =
   \begin{bmatrix}
   1 & 0 & t_x \\
   0 & 1 & t_y \\
   0 & 0 & 1
   \end{bmatrix}
   \]

\subsection*{Affine Transformations}
1. Combine linear transformations (scaling, rotation, shear) and translation:
   \[
   \mathbf{A} =
   \begin{bmatrix}
   a_{11} & a_{12} & t_x \\
   a_{21} & a_{22} & t_y \\
   0 & 0 & 1
   \end{bmatrix}
   \]
2. Parallel lines remain parallel under affine transformations.

\subsection*{Using RANSAC for Robust Fitting}
1. RANSAC (Random Sample Consensus):
   - Randomly select a set of pts, estimate the transformation. \\
   - Compute the transformation and identify inliers. \\
   - If enough inliers, recompute the transformation with all inliers. \\
2. Keep the transformation with the most inliers across multiple trials.

\subsection*{Summary of RANSAC}
1. \textbf{Pros}: \\
   - Robust to noise \& outliers(extremely x). \\
   % - Works well for a variety of problems. \\
2. \textbf{Cons}: \\
   - Requires careful tuning of hyperparams. \\
   - Perf drops with low inlier ratios. \\
   - Can struggle with poor initialization based on minimum samples.
