\section{Local Feature Description}

% \subsection*{Local Feature Detection and Description}
% 1. Detection: Identify interest points using methods like the Harris Corner Detector or Blob Detector. \\
% 2. Description: Extract vector feature descriptors surrounding each interest point. \\
% 3. Matching: Determine correspondences between descriptors in different views.

\subsection*{SIFT Descriptor (Lowe, 2004)}
1. Compute gradients within sub-patches --> histograms of gradient orientations. \\
2. Rotate the patch based on dominant gradient --> rotation invariance. \\
3. Map pixels to a 128-dimensional vec. \\
4. Invariance to scale and rotation, partial invariance to illumination changes, capable of handling occlusion.

\subsection*{Blob Detection and Difference of Gaussians (DoG)}
1. Key: identifying maxima or minima in both position and scale. \\
2. The Laplacian of Gaussian (LoG) can be approximated with DoG for better efficiency:
   \[
   \text{DoG}(x, y, \sigma) = G(x, y, k\sigma) - G(x, y, \sigma)
   \]
3. The characteristic scale corresponds to the peak LoG response, capturing feature blobs across multiple scales.

\subsection*{Matching Techniques}
1. Compute candidate matches w Sum of Squared Distances (SSD) etc. \\
2. Use Nearest Neighbor Search with a thres of nearest to second-nearest descriptor. \\
   \[
   \text{Ratio} = \frac{\text{Distance to best match}}{\text{Distance to second-best match}}
   \]
   If the ratio is low, the match is reliable; if high, it may indicate ambiguity.

% \subsection*{Applications of Local Invariant Features}
% 1. Wide baseline stereo matching. \\
% 2. Motion tracking and panoramas. \\
% 3. Mobile robot navigation and 3D reconstruction. \\
% 4. Object and scene recognition.

