\section{Stereo and Camera Calibration}
% 1. \textbf{Stereo Vision:} recover the 3D structure of a scene from multiple 2D images. \\
% 2. \textbf{Single-View Limitation:} depth information is ambiguous as each pixel only provides a ray. \\

\subsection*{Stereo Vision}
1. \textbf{Stereo Vision Setup:} Two or more cams capture the scene from slightly different viewpoints. \\
% 2. \textbf{Correspondences:} Matching points (features) in the images are identified to calculate depth. \\
% 3. \textbf{Epipolar Geometry:} A key concept in stereo vision that defines the relationship between two views. \\
2. \textbf{Disparity:} The difference in the position of corresponding pts between the two images. Depth is inversely proportional to disparity:
   \[
   \text{Depth} = \frac{f \cdot B}{\text{Disparity}}
   \]
   where \(f\) is the focal length, and \(B\) is the baseline (distance between the cameras). \\



\subsection*{Projection Matrix and Calibration Process}
\textbf{Projection Model:} 
% 1. The full projection matrix relates world coordinates to image coordinates:
   % \[
   % p = MX, M = K [R | t]
   % \]
   \[
   \mathbf{p} = \lambda M \mathbf{X}
   \]
   where \(M = K [R | t]\), \(3 \times 4\) projection matrix \\
% 1. \textbf{Camera calibration:} Estimates the intrinsic matrix \(K\) and extrinsic parameters \([R | t]\) to convert 3D points to 2D image coordinates.   \\
1. \textbf{Camera calibration:} Estimates M given p, X.   \\
- \textbf{Calibration Targets:} Objects with known dimensions (like checkerboards) \\
% are used to find correspondences and solve for the calibration matrix. \\
- \textbf{Linear Calibration Method:} Uses correspondences between 3D points and their projections to estimate \(M\). \\
   % - \textbf{Projection Model:} 
   %   \[
   %   \mathbf{p} = \lambda M \mathbf{X},
   %   \]
   - \textbf{Linearization via Cross-Product:} The projection model is linearized as:
     \[
     \mathbf{p} \times (M \mathbf{X}) = 0.
     \]
       \[
       \begin{bmatrix}
       u_i \\
       v_i \\
       1
       \end{bmatrix}
       \times
       \begin{bmatrix}
       m_1^T \mathbf{X}_i \\
       m_2^T \mathbf{X}_i \\
       m_3^T \mathbf{X}_i
       \end{bmatrix}
       =
       \begin{bmatrix}
       0 \\
       0 \\
       0
       \end{bmatrix}.
       \]
    \[
   \begin{bmatrix}
       0^T & -\mathbf{X}_i^T & v_i \mathbf{X}_i^T \\
       \mathbf{X}_i^T & 0^T & -u_i \mathbf{X}_i^T \\
       -v_i \mathbf{X}_i^T & u_i \mathbf{X}_i^T & 0^T
       \end{bmatrix}
       \begin{bmatrix}
       m_1 \\
       m_2 \\
       m_3
       \end{bmatrix}
       =
       \begin{bmatrix}
       0 \\
       0 \\
       0
       \end{bmatrix}.
    \]
    \textbf{Note: }2 linearly independent eqs per correspondence, M (3x4) 11 deg of freedom\\
   - \textbf{Stacking Equations:}  For \(n\) correspondences:
      \[
       A \mathbf{m} = 0,
   \]
   where \(A\) is a \(2n \times 12\) matrix and \(\mathbf{m}\) contains 12 unknown entries of \(M\).

   - \textbf{Solving the System:} Use SVD to find the eigenvector of \(A^T A\) corresponding to the smallest eigenvalue, providing an initial estimate of \(M\). \\
   % - \textbf{Refinement with Non-linear Optimization:} A non-linear optimizer (e.g., Levenberg-Marquardt) refines the solution to handle noise and improve accuracy. \\
  - Note: Can use a\textbf{ non-linear optimizer} (e.g., Levenberg-Marquardt) to handle noise and improve accuracy. \\

2. \textbf{Triangulation:} Given M,p --> X\\
% Given multiple images and correspondences, triangulation computes the 3D location of a point by finding the intersection of rays from each camera. \\
   - \textbf{Method 1: Geometric Approach}  
     Find the shortest segment between viewing rays and select the midpoint of this segment. \\
   - \textbf{Method 2: Non-linear Optimization}  
     % Minimize the reprojection error by finding the point \(X\) that minimizes:
     \[
     X = argmin (
     \sum_i d(\mathbf{p}_i, M_i X)^2 ),
     \]
% \subsection*{Summary}
% 1. \textbf{Stereo vision} recovers depth by matching points across multiple images. \\
% 2. \textbf{Camera calibration} finds the intrinsic and extrinsic parameters to project 3D points into 2D. \\
% 3. Understanding these principles is essential for tasks such as depth estimation, 3D reconstruction, and structure from motion.
