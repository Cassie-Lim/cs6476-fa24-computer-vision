\section{Fitting and Hough Transform}
% 1. Goal: Find the best parametric model (e.g., line, circle) to represent a set of features in images, despite noise, missing parts, or clutter. \\
% 2. Challenges in Line Fitting: \\
%    - Extra edge points (clutter) may not belong to any line. \\
%    - Incomplete lines must be bridged from partial evidence. \\
%    - Noise in edge points and orientations complicates accurate fitting. \\

\subsection*{Hough Transform for Line Detection}
% 1. Concept: Use a \textbf{voting mechanism} to let each feature vote for possible model parameters.  \\
1. Mapping to Hough Space: \\
   - \textbf{Image space} $(x, y)$: Holds edge points. \\
   - \textbf{Hough space} $(m, b)$: Lines $y = mx + b$. \\
   - A point in the image space --> a line in Hough space. \\
2. Advantages: robust to noise, tolerance to disconnected segments
\subsection*{Hough Transform Algorithm (Line Detection)}
1. Initialize accumulator arr $H[m, b] = 0$. \\
2. For each edge point $(x, y)$ in the image: \\
   \null \quad For each possible slope $m$: \\
   \null \quad \quad Calculate $b = y - mx$. \\
   \null \quad \quad Increment $H[m, b]$ by 1. \\
3. Find the $(m, b)$ with the highest votes in $H$. \\
% 4. The detected lines correspond to the parameters with maximum votes.

\subsection*{Polar Representation of Lines}
1. To avoid infinite slopes for vertical lines, use the polar equation: \\
   \[
   x \cos \theta + y \sin \theta = d
   \]
   % where $d$ is the perpendicular distance from the origin, and $\theta$ is the angle between the x-axis and the perpendicular to the line. \\
2. Each edge point votes for a sinusoid in $(d, \theta)$ space, and intersections correspond to lines in the original image space.
\subsection*{Extensions of Hough Transform}
- Use gradient direction to reduce param search. \\
- Assign higher weights to stronger edges during voting. \\
- Adapt the transform for circles etc. \\

% \subsection*{Extensions and Applications}
% 1. Extensions: \\
%    - Use gradient direction to reduce parameter search. \\
%    - Assign higher weights to stronger edges during voting. \\
%    - Adapt the transform for circles or other shapes. \\
% 2. Applications: \\
%    - Line and circle detection in images. \\
%    - Iris detection and coin recognition. \\
%    - Generalized Hough Transform for arbitrary shape matching.

\subsection*{Pros and Cons of Hough Transform}
1. \textbf{Pros:} \\
   - Can handle occlusion, noise, and gaps in features. \\
   - Detects multiple instances of a model in a single pass. \\
2. \textbf{Cons:} \\
   - Search time increases exponentially with more parameters. \\
   - Non-target shapes may produce spurious peaks. \\
   - Requires careful choice of grid size for parameter space.
