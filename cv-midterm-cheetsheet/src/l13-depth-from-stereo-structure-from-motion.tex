\section{Depth from Stereo \& Structure from Motion}

\subsection*{Depth from Stereo:}
1. \textbf{Goal:} Recover depth by finding corresponding points in two images (stereo matching). Depth is inversely related to disparity.
   
2. \textbf{Disparity:} 
   - Disparity \( d = x - x' \), where \( x \) and \( x' \) are corresponding image coordinates in two views.
   % - Depth \( z \) is related to disparity as:
     \[
     depth = \frac{f B}{d},
     \]
     \(d\): depth, \( f \) : focal length, \( B \) : baseline distance between cameras.

3. \textbf{Stereo Matching Algorithm:}
   - For each pixel in imgA, find corresponding epipolar line in imgB.
   - Search along epipolar line and select  best match based on a similarity measure (e.g., SSD, normalized correlation).
   - Triangulate to obtain depth.

4. \textbf{Rectification:} Reproject image planes onto a common plane to transform epipolar lines into horizontal scanlines, simplifying correspondence search.

5. \textbf{Basic Challenges:}
   - Textureless regions: Hard to find unique correspondences.
   - Repeated patterns: Ambiguity in matching points.
   - Specular surfaces: Appearance changes with viewpoint.

\subsection*{Structure from Motion (SfM):}
1. \textbf{Goal:} Recover 3D structure and camera motion from multiple views.

2. \textbf{Projection Model:}
   - For a 3D point \( X_j \) and camera \( M_i \), the image point \( p_{ij} \) is given by:
     \[
     p_{ij} \equiv M_i X_j, \quad p_{ij} = M_i X_j.
     \]
   - \( M_i = K_i [R_i | t_i] \), \( K_i \) : intrinsic matrix,  \( R_i, t_i \): extrinsic parameters.

3. \textbf{Ambiguities in SfM:}
   - \textbf{Scale ambiguity:} Cannot determine absolute scale from motion alone.
   - \textbf{Projective ambiguity:} Without constraints, reconstruction is only determined up to a projective transformation.
   - \textbf{Affine ambiguity:} If parallel lines are known, ambiguity reduces to affine. Additional constraints can reduce ambiguity to similarity.

4. \textbf{Affine Structure from Motion:}
   - Use affine cameras for a simplified SfM approach. For \( m \) cameras and \( n \) points:
     \[
     p_{ij} \equiv A_i X_j + b_i,
     \]
      \( A_i \) : affine projection matrix. \( b_i \) :  translation.\\
   - Factorize the measurement matrix to recover motion and structure:
     \[
     D = \begin{bmatrix} A_1 \\ \vdots \\ A_m \end{bmatrix} \begin{bmatrix} X_1 & \dots & X_n \end{bmatrix}.
     \]

5. \textbf{Triangulation in SfM:}
   - Estimate the 3D points \( X_j \) by minimizing the reprojection error across all views:
     \[
     \sum_i \| p_{ij} - M_i X_j \|^2.
     \]
   - Use optimization techniques such as bundle adjustment to refine the structure and motion estimates.

% \subsection*{Summary:}
% - \textbf{Depth from Stereo:} Find correspondences between two images, compute disparity, and infer depth.
% - \textbf{Structure from Motion:} Recover 3D structure and camera motion from multiple views, with challenges in handling ambiguities.
